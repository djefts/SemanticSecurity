\documentclass[conference]{IEEEtran}
\IEEEoverridecommandlockouts
\usepackage{cite}
\usepackage{amsmath,amssymb,amsfonts}
\usepackage{algorithmic}
\usepackage{graphicx}
\usepackage{textcomp}
\usepackage{xcolor}
\usepackage{float}
\def\BibTeX{{\rm B\kern-.05em{\sc i\kern-.025em b}\kern-.08em
    T\kern-.1667em\lower.7ex\hbox{E}\kern-.125emX}}
\begin{document}

\title{Leveraging Web User Information to Generate Ontologies
}

\author{\IEEEauthorblockN{David Jefts}
\IEEEauthorblockA{\textit{Software Engineering} \\
\textit{Embry-Riddle Aeronautical University}\\
Daytona Beach, United States}
\and
\IEEEauthorblockN{Juan Ortiz Couder}
\IEEEauthorblockA{\textit{Software Engineering} \\
\textit{Embry-Riddle Aeronautical University}\\
Daytona Beach, United States}
}

\maketitle

\begin{abstract}
As people are continually more active in social media, a plethora of information from individual users is released into the internet and, in most cases, discoverable by nearly any other user. This poses a problem because many account-recovery mechanisms depend on security questions to validate a user and hackers can infer a security question answer through an analysis of available information in social media. To demonstrate the amount of information that can be extracted from a user’s account, this paper describes an approach to extract and analyze the information available in their social media. The user’s social media account information is collected to create verified links between the user and their online profiles, then scraped their account for all available data. The user’s social media information is parsed using various Natural Language Processing techniques to extract the machine-relevant information. An ontology group was developed to describe behavioral, social, physical, and ideological relationships and populate its corresponding Knowledge Graph. Because the information in a Knowledge Graph is easily accessed by both humans and machines, can have mathematical Graph Theories applied to it, and is easily transferable, it can be used in a wide variety of applications. 
\end{abstract}

\begin{IEEEkeywords}
component, formatting, style, styling, insert
\end{IEEEkeywords}

\section{Introduction}
With the advent of social media, and its increased accessibility to what used to be considered private information, there is a need to improve the security of account recovery mechanisms. Some examples of account recovery mechanisms are URLs sent through email, or PINs the user determined when creating the account, or Backup Codes – similar to PINs but longer and, the use of security questions. Common security questions include: “In what city were you born?”, “What is the name of your favorite pet?”, “What is your mother’s maiden name?”, “What high school did you attend?” As an example, consider you are a person that wants to access another users’ Facebook account but you don’t know the users’ account password. You could investigate that users’ account, identify who their mother is, then look for their parents’ last name and you could find the users’ maiden name. If you wanted to find the user’s high school, you could look through their pictures and look for graduation photos where you could see the high school’s name. Some users even put the high school where they attended in their biography section; the same goes with the city in which they were born. Clearly, using the information available in a social network can allow for an unauthorized person to deduce the information to answer an account’s security questions. It has been estimated that cybercrime will cost the world 10.5 trillion annually by 2025. In other words, that’s approximately 1,350 per person, every year. If security questions are improved, it could help to reduce that amount considerably.  
	This work aims to develop a software system capable of going through a user’s social media account and find what security questions can be answered from the information extracted. Firstly, the user’s account will be parsed and stored in a Knowledge Graph. This Knowledge Graph will be based on Ontologies that describe social and physical relationships. Then, the information stored in the Knowledge Graph will be searched to identify the user’s exposure of sensitive information that may be used to answer the account recovery security questions. Moreover, this system could potentially generate a new set of security questions that cannot be easily answered by using the information that the user has already posted. 


\section{Methodology}
A scraper was developed to extract information. This scraper was applied to the social media account of a user, who granted us access to their account. The information gathered included the biography section of their profile page. This information contained degrees, schools, hometown, pronunciation of their name, and where the user currently lives in. Additional information collected included the hobbies and the user’s friends. All this is done through the tags Facebook uses to identify the information. Lastly, the posts made by the user are parsed and stored at first to be later processed.
Once all the information from the users account has been scraped, different entities are extracted and identified. This is accomplished using Natural Language Toolkit (NLTK) to determine the different parts of speech in the information that was retrieved from Facebook. During this entity extraction, the words in every sentence are classified (adjectives, nouns, verbs, pronouns…) and relationships between the words are determined. For instance, if a user named Tim posts “Just had donuts for the first time, they were great”, the words “donuts” and “great” would tell us what Tim thinks about donuts. 
The next step is to model all this data into a knowledge graph using ontologies. The ontologies used to describe the relationships between users and the data retrieved from Facebook are Friend Of A Friend (FOAF) and Semantically-Interlinked Online Communities (SIOC). FOAF is an existing ontology that describes persons, their activities and relations to other persons and objects. SIOC is another existing ontology used to describe a person’s information stored in internet discussion methods such as blogs, forums, and mailing list among others. All the information sorted in the ontologies is to be added to knowledge graphs. In the knowledge graph, each person that has any relationship with the user is a different node. After this, the user hobbies along with the users’ basic information were also added to the graph to populate the main user node.   
 
\begin{figure}[htbp]
\centerline{\includegraphics[width=0.5\textwidth]{./DFD.jpg}}
\caption{Data Flow Diagram}
\label{fig}
\end{figure}

This diagram shows the data flow of the system. As explained in the previous paragraphs, the information comes from Facebook and then is parsed through the tag and post information, which will later be used to extract the entities and processed to create the ontology with the different relationships those entities have. Finally, the ontology will be used to generate the nodes that will be used to populate the knowledge graph.

\section{Results}
The previously stated method resulted in the creation of a Knowledge graph displayed in the figure below. 

\begin{figure}[H]
\centerline{\includegraphics[width=0.5\textwidth]{./Knowledge_Graph_1.png}}
\caption{Knowledge Graph}
\label{fig1}
\end{figure}

In this Knowledge Graph, it can be seen four different types of nodes. The nodes in red show the different people the user interacts with, in this case, friends in the social media. The nodes in blue are the different entities extracted from the user posts. The node in lighter blue displays the user id from the social network. Finally, the nodes in yellow are the id of the interaction where the other nodes are extracted from. The ontologies can extract most of the useful information from the posts, however it can be seen that it does struggle with words that start in an upper-case letter. This is because it believes that capitalized words are important.
	The following screenshot displays in more detail the knowledge graph. 

\begin{figure}[H]
\centerline{\includegraphics[width=0.5\textwidth]{./Knowledge_Graph_2.png}}
\caption{Knowledge Graph}
\label{fig2}
\end{figure}

The node expanded is the one with the circle around. In the right side of the screenshot, it is shown the post id, the type the post was detected as for the ontologies used, in this case the post is a “Post” type for the “soic” ontology and “Document” for the “ns1” document. It is also displayed comment from the post, and the date the post was created. 

\section{Conclusion}


\section{Further Works}
Based on what it has been accomplished by creating knowledge graphs from information posted on social media accounts, further work could lead towards the notification to the users whos’ information posted online is related to their security questions. Once this is accomplished and proven to be successful, it could be expanded to other social media sites (Instagram, Twitter, Reddit…). 
	Since the current project only allows the use of textual data, in the future, next iterations of this project could be improved to allow the use of pictures or videos. For people who have posts online related to them (interviews, articles about them…), this information could also be used to gain access to their accounts. 
	Finally, another possibility for further works, would be to evaluate the likelihood to access a user’s account with the current information posted, not just if it is possible or not, but how likely it is, this could also be added to the knowledge graphs as nodes.


\section{References}

These are only placeholders

\begin{thebibliography}{00}
\bibitem{b1} G. Eason, B. Noble, and I. N. Sneddon, ``On certain integrals of Lipschitz-Hankel type involving products of Bessel functions,'' Phil. Trans. Roy. Soc. London, vol. A247, pp. 529--551, April 1955.
\bibitem{b2} J. Clerk Maxwell, A Treatise on Electricity and Magnetism, 3rd ed., vol. 2. Oxford: Clarendon, 1892, pp.68--73.
\bibitem{b3} I. S. Jacobs and C. P. Bean, ``Fine particles, thin films and exchange anisotropy,'' in Magnetism, vol. III, G. T. Rado and H. Suhl, Eds. New York: Academic, 1963, pp. 271--350.
\bibitem{b4} K. Elissa, ``Title of paper if known,'' unpublished.
\bibitem{b5} R. Nicole, ``Title of paper with only first word capitalized,'' J. Name Stand. Abbrev., in press.
\bibitem{b6} Y. Yorozu, M. Hirano, K. Oka, and Y. Tagawa, ``Electron spectroscopy studies on magneto-optical media and plastic substrate interface,'' IEEE Transl. J. Magn. Japan, vol. 2, pp. 740--741, August 1987 [Digests 9th Annual Conf. Magnetics Japan, p. 301, 1982].
\bibitem{b7} M. Young, The Technical Writer's Handbook. Mill Valley, CA: University Science, 1989.
\end{thebibliography}
\vspace{12pt}

\end{document}
